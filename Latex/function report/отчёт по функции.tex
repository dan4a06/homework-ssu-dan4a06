\input{C:/Latex/preamble.tex}
\newcommand{\n}{\par}
\usepackage{ dsfont }
\pgfplotsset{width=13cm,compat=1.15}

\begin{document}
	\title{Доклад по функциям}
	\author{Афанасьев Даниил}
	\date{\today}
	
	\maketitle
	
	\newpage
	
	\tableofcontents
	
	\newpage
	\centering
	
	%\section{Функция y = [x]}
	
	\begin{flushleft}
		
		\section{Функция y = [x]}
		Функция $y=[x]$ обозначает функцию целой части (или функцию взятия целой части числа). Эта функция возвращает наибольшее целое число, которое меньше или равно $x$. \n
		Математически это записывается так:
		\[y=[x]=max\{k\in\mathds{Z} \mid \hspace{1mm} \leq x\}.\]
		
		\subsection*{Область определения}
		Функция $y=[x]$ определена для всех действительных чисел:
		\[D(y)=\mathds{R}.\]
		
		\subsection*{Нули функции}
		Нули функции — это значения $x$, при которых $y=0$. Целая часть числа $x$ равна нулю на полуинтервале $[0,1)$:
		\[y=0 \text{ при } x\in[0,1).\]
		
		\subsection*{Чётность\textbackslash нечётность}
		Функция $y=[x]$ не является ни чётной, ни нечётной.
		
		\begin{itemize}
			\item $y(-x) = [-x]$.
			\item $ y(-x)\neq y(-x) \text{ и } y(-x) \neq -y(x)$.
		\end{itemize}
		
		\subsection*{Периодичность}
		Функция $y=[x]$ не является периодической, так как её значения изменяются скачкообразно при переходе через целые числа.
		
		\subsection*{Монотонность}
		Функция $y=[x]$ кусочно-постоянна на полуинтервалах $[n,n+1)$, где $n\in\mathds{Z}$ Производная на этих интервалах:
		\[y'=(x)=0\text{ для }x\in(n,n+1).\]
		В точках $x=n$ производная не существует.
		
		\subsection*{Точки экстремума}
		Функция $y=[x]$ не имеет точек экстремума, так как она кусочно-постоянна.
		
		\subsection*{Максимум и минимум функции}
		Функция $y=[x]$ не имеет максимума и минимума, так как её значения не ограничены сверху или снизу.
		
		\subsection*{Выпуклость вверх\textbackslash вниз, точки перегиба}
		Функция $y=[x]$ является кусочно-постоянной, и её график состоит из горизонтальных отрезков. На интервалах $(n,n+1)$ функция линейна(постоянная), поэтому:
		
		\begin{itemize}
			\item Выпуклость вверх или вниз не определена.
			\item Точек перегиба нет.
		\end{itemize}
		
		\subsection*{Асимптоты}
		
		\begin{itemize}
			\item Вертикальные асимптоты: отсутствуют.
			
			\item Горизонтальные асимптоты: отсутствуют.
			
			\item Наклонные асимптоты: отсутствуют.
		\end{itemize}
		
		\subsection*{Множество значений}
		Функция $y=[x]$ принимает все целые значения:
		\[E(x)=\mathds{Z}.\]
		 
		\subsection*{График функции}
		 
		\begin{center}
		\begin{tikzpicture}
				\begin{axis}[
					axis lines=middle,
					xlabel={$x$},
					ylabel={$y$},
					xmin=-5, xmax=6,
					ymin=-5, ymax=6,
					xtick={-5,-4,...,5},
					ytick={-5,-4,...,5},
					samples=100,
					domain=-5:5,
					restrict y to domain=-5:5,
					]
					
					\addplot[blue, thick, jump mark left] {floor(x)};
				\end{axis}
		\end{tikzpicture}
		\end{center}		 
		%===============================================
		 
		\section{Функция y = \{x\}}
		Функция $y={x}$ обозначает дробную часть числа  x. Она определяется как разность между числом x и его целой частью $[x]$:
		\[{x}=x−[x].\]
		 
		\subsection*{Область определения}
		Функция $y={x}$ определена для всех действительных чисел:
		\[D(y)=\mathds{R}.\]
		 
		\subsection*{Нули функции}
		Нули функции — это значения $x$, при которых $y=0$. Дробная часть числа $x$ равна нулю, когда $x$ --- целое число:
		\[y=0 \text{ при } x\in\mathds{Z}.\]
		 
		\subsection*{Чётность\textbackslash нечётность}
		Функция $y={x}$ является чётной, так как:
		\[\{−x\}=\{x\}\text{ для всех }x\in\mathds{R}.\]
		 	 
		\subsection*{Периодичность}
		Функция $y={x}$ является периодической с периодом $T=1$, так как:
		\[\{x+1\}=\{x\}\text{ для всех } x\in\mathds{R}.\]
		 
		\subsection*{Монотонность}
		Функция $y={x}$ линейна на интервалах $(n,n+1)$, где $n\in\mathds{Z}$. Производная на этих интервалах:
		\[y'(x)=1 \text{ для }x\in(n,n+1).\]
		В точках $x=n$ производная не существует.
		 
		\subsection*{Точки экстремума}
		Функция $y={x}$ не имеет точек экстремума, так как она линейно возрастает на интервалах $(n,n+1)$
		
		\subsection*{Максимум и минимум функции}
		 
		\begin{itemize}
			\item Минимум: $y=0$ достигается при всех целых значениях $x$ (т.е. $x\in\mathds{Z}$).
			\item Максимум: y стремится к 1, но никогда не достигает её. Таким образом, точной верхней границей является 1, но максимум в строгом смысле отсутствует.
		\end{itemize}
		 
		\subsection*{Выпуклость вверх\textbackslash вниз, точки перегиба}
		Функция $y={x}$ линейна на интервалах $(n,n+1)$, поэтому:
		 
		\begin{itemize}
			\item Выпуклость вверх или вниз не определена.
			\item Точек перегиба нет.
		\end{itemize}
		 
		\subsection*{Асимптоты}
		 
		\begin{itemize}
			\item Вертикальные асимптоты: отсутствуют.
		 	
		 	\item Горизонтальные асимптоты: отсутствуют.
		 	
		 	\item Наклонные асимптоты: отсутствуют.
		\end{itemize}
		 
		\subsection*{Множество значений}
		 Функция $y={x}$ принимает значения на полуинтервале $[0,1)$:
		 \[E(x)=[0,1).\]
		 
		\subsection*{График функции}
		
		 \begin{center}
		 \begin{tikzpicture}
		 	\begin{axis}[
		 		axis lines=middle,
		 		xlabel={$x$},
		 		ylabel={$y$},
		 		xmin=-5,xmax=6,
		 		ymin=-2,ymax=2.5,
		 		xtick={-5,-4,...,5},
		 		ytick={-2,-1.5,...,2},
		 		samples=1000,
		 		domain=-5:5,
		 		restrict y to domain=-3:3,
		 		]
		 		\addplot[blue, thick, jump mark right]{x - floor(x)};
		 	\end{axis}
		 \end{tikzpicture}
		 \end{center}
		 
		%===============================================
		 
		\section{Функция y = sign(x)}
		Функция $y=sign(x)$ (или сигнум-функция) определяет знак числа $x$. Она возвращает:
		 
		\[
		sign(x)=
		\begin{cases}
			-1, & \text{если } x < 0,\\
			 0, & \text{если } x = 0,\\
			 1, & \text{если } x > 0.
		\end{cases}
		\]
		 
		\subsection*{Область определения}
		Функция $y=sign(x)$ определена для всех действительных чисел:
		\[D(y)=\mathds{R}.\]
		 
		\subsection*{Нули функции}
		Нули функции — это значения $x$, при которых $y=0$:
		\[y=0 \text{ при } x=0.\]
		 
		\subsection*{Чётность\textbackslash нечётность}
		Функция $y=sign(x)$ является нечётной, так как:
		\[sign(-x)=-sign(x) \text{ для всех }x\in\mathds{R}.\]
		 
		\subsection*{Периодичность}
		Функция $y=sign(x)$ не является периодической.
		 
		\subsection*{Монотонность}
		Функция $y=sign(x)$ постоянна на интервалах $(−\infty,0)$ и $(0,+\infty)$. Производная на этих интервалах:
		\[y′(x)=0\text{ для }x\neq0.\]
		В точке $x=0$ производная не существует.
		 
		\subsection*{Точки экстремума}
		Функция $y=sign(x)$ не имеет точек экстремума, так как она постоянна на интервалах $(−\infty,0)$ и $(0,+\infty)$.
		
		\subsection*{Точки разрыва}
		Функция $y=sign(x)$ имеет точку разрыва первого рода в $x=0$. В этой точке:
		
		\begin{itemize}
			\item Левый предел:$lim_{x\rightarrow-0}sign(x)=-1$
			\item Правый предел: $lim_{x\rightarrow+0}sign(x)=1$
			\item Значение функции в точке $x=0:sign(0)=0$.
		\end{itemize}
		
		Таким образом, в точке x=0 функция имеет скачок.
		 
		\subsection*{Максимум и минимум функции}
		 
		Функция $y=sign(x)$ имеет максимум $y=1$ при $x>0$ и минимум $y=−1$ при $x<0$.
		 
		\subsection*{Выпуклость вверх\textbackslash вниз, точки перегиба}
		Функция $y=sign(x)$ постоянна на интервалах $(−\infty,0)$ и $(0,+\infty)$,поэтому:
		 
		\begin{itemize}
		 	\item Выпуклость вверх или вниз не определена.
		 	\item Точек перегиба нет.
		\end{itemize}
		 
		\subsection*{Асимптоты}
		 
		\begin{itemize}
			\item Вертикальные асимптоты: отсутствуют.
		 	
			\item Горизонтальные асимптоты: отсутствуют.
		 	
		 	\item Наклонные асимптоты: отсутствуют.
		\end{itemize}
		 
		\subsection*{Множество значений}
		Функция $y=sign(x)$ принимает три значения:
		\[E(y)=\{−1,0,1\}.\]
		
		\subsection*{График функции}
		 
		  \begin{center}
		 	\begin{tikzpicture}
		 		\begin{axis}[
		 			axis lines=middle,
		 			xlabel={$x$},
		 			ylabel={$y$},
		 			xmin=-3, xmax=3,
		 			ymin=-2, ymax=2,
		 			xtick={-3,-2,...,3},
		 			ytick={-1,0,1},
		 			samples=100, % Количество точек для построения
		 			domain=-3:3,
		 			]
		 			
		 			\addplot[blue, thick, jump mark left] {sign(x)};
		 			\addplot[blue, mark=*, only marks] coordinates {(0, 0)};
		 		\end{axis}
		 	\end{tikzpicture}
		 \end{center}
		 
		\end{flushleft}
	
\end{document}