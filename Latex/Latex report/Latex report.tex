\documentclass[a4paper, 14pt]{extarticle}

%Поддержка языков
\usepackage[english, russian]{babel}

%Найстройка кодировок
\usepackage[T2A]{fontenc}
\usepackage[utf8]{inputenc}

%Настройка шрифта
\usepackage{fontspec}
\setmainfont[Ligatures=TeX]{Times New Roman} % Шрифт для основного текста документа
\setsansfont[Ligatures=TeX]{Arial}
\setmonofont{Consolas} % Шрифт для кода

% Настройка отступов от краев страницы
\usepackage[left=3cm, right=1.5cm, top=2cm, bottom=2cm]{geometry}

\usepackage{titleps} % Колонтитулы
\usepackage{subfig} % Для подписей к рисункам и таблицам
\usepackage{graphicx} % для вставки картинок

% Пакет для отрисовки графиков
\usepackage{tikz}
\usetikzlibrary{arrows,positioning,shadows}
\usepackage{pgfplots}
\pgfplotsset{width=10cm, compat=1.9}

\usepackage{stmaryrd} % Стрелки в формулах
\usepackage{indentfirst} % Красная строка после заголовка
\usepackage{hhline} % Улучшенные горизонтальные линии в таблицах
\usepackage{multirow} % Ячейки в несколько строчек в таблицах
\usepackage{longtable} % Многостраничные таблицы
\usepackage{paralist,array} % Список внутри таблицы
\usepackage[normalem]{ulem}  % Зачеркнутый текст
\usepackage{upgreek, tipa} % Красивые греческие буквы
\usepackage{amsmath, amsfonts, amssymb, amsthm, mathtools} % ams пакеты для математики, табуляции
\usepackage{nicematrix} % Особые матрицы pNiceArray

\linespread{1.5} % Межстрочный интервал
\setlength{\parindent}{1.25cm} % Табуляция
\setlength{\parskip}{0cm}

% Пакет для красивого выделения кода
\usepackage{listings}
\usepackage{xcolor}

% Настройка для отображения кода LaTeX
\lstset{
	language=[LaTeX]TeX, % Язык — LaTeX
	basicstyle=\ttfamily\small, % Шрифт и размер текста
	keywordstyle=\color{blue}, % Цвет ключевых слов
	commentstyle=\color{green}, % Цвет комментариев
	stringstyle=\color{red}, % Цвет строк
	numbers=left, % Нумерация строк слева
	numberstyle=\tiny\color{gray}, % Стиль нумерации строк
	stepnumber=1, % Шаг нумерации строк
	numbersep=5pt, % Отступ номера строк от кода
	backgroundcolor=\color{white}, % Цвет фона
	showspaces=false, % Показывать пробелы
	showstringspaces=false, % Показывать пробелы в строках
	showtabs=false, % Показывать табуляции
	frame=single, % Рамка вокруг кода
	rulecolor=\color{black}, % Цвет рамки
	tabsize=4, % Размер табуляции
	captionpos=b, % Позиция заголовка (b — снизу)
	breaklines=true, % Переносить длинные строки
	breakatwhitespace=true, % Переносить только по пробелам
	title=\lstname % Заголовок (имя файла)
}

% Добавляем гипертекстовое оглавление в PDF
\usepackage[
bookmarks=true, colorlinks=true, unicode=true,
urlcolor=black,linkcolor=black, anchorcolor=black,
citecolor=black, menucolor=black, filecolor=black,
]{hyperref}

% Убрать переносы слов
\tolerance=1
\emergencystretch=\maxdimen
\hyphenpenalty=10000
\hbadness=10000

\newpagestyle{main}{
	% Верхний колонтитул
	\setheadrule{0cm} % Размер линии отделяющей колонтитул от страницы
	\sethead{}{}{} % Содержание {слева}{по центру}{справа}
	% Нижний колонтитул
	\setfootrule{0cm} % Размер линии отделяющей колонтитул от страницы
	\setfoot{}{}{\thepage} % Содержание {слева}{по центру}{справа}
}
\pagestyle{main}



\begin{document}
	\title{Отчет по изучению \LaTeX{}}
	\author{Афанасьев Даниил}
	\date{\today}
	
	\maketitle
	
	\newpage
	
	\tableofcontents
	
	\newpage
	\centering
	
	\section{Для чего \LaTeX}
	
	\subsection{Для чего используется \LaTeX}
	\LaTeX — это система компьютерной вёрстки, которая широко используется для создания научных, технических и академических документов. Она особенно популярна в математике, физике, информатике и других науках, где требуется высококачественное оформление формул, таблиц и графиков. Основные преимущества \LaTeX включают:
	
	\begin{itemize}
		\item \textbf{Высокое качество типографики}: \LaTeX автоматически управляет шрифтами, отступами, межстрочными интервалами и другими аспектами вёрстки, что позволяет создавать профессионально выглядящие документы.
		\item \textbf{Удобство работы с математическими формулами}: \LaTeX предоставляет мощные инструменты для набора сложных математических выражений.
		\item \textbf{Кроссплатформенность}: Документы, созданные в \LaTeX, могут быть скомпилированы на любой операционной системе (Windows, macOS, Linux).
		\item \textbf{Разделение содержания и оформления}: Автор может сосредоточиться на содержании, а \LaTeX позаботится о внешнем виде документа.
		\item \textbf{Поддержка больших документов}: \LaTeX идеально подходит для создания книг, диссертаций и других объёмных работ.
	\end{itemize}
	
	\newpage
	
	\subsection{История \LaTeX}
	\LaTeX был создан Лесли Лэмпортом в 1984 году как надстройка над системой TeX, разработанной Дональдом Кнутом в 1978 году. TeX был создан для того, чтобы автор мог сосредоточиться на содержании, а не на оформлении документа. \LaTeX упростил использование TeX, добавив множество макросов и шаблонов.
	
	\begin{itemize}
		\item \textbf{1984 год}: Лесли Лэмпорт выпустил первую версию \LaTeX.
		\item \textbf{1994 год}: Вышла версия \LaTeX2e, которая до сих пор является стандартом.
		\item \textbf{Современность}: Сегодня \LaTeX активно развивается, и сообщество пользователей продолжает создавать новые пакеты и расширения.
	\end{itemize}
	
	\subsection{Как использовать \LaTeX}
	Для использования \LaTeX необходимо установить дистрибутив, например, TeX Live (для Linux и macOS) или MiKTeX (для Windows). После установки можно писать код в любом текстовом редакторе и компилировать его в PDF. Популярные редакторы:
	
	\begin{itemize}
		\item \textbf{Overleaf}: Онлайн-редактор, который не требует установки и позволяет работать в команде.
		\item \textbf{TeXShop}: Редактор для macOS.
		\item \textbf{TeXworks}: Простой редактор для Windows и Linux.
	\end{itemize}
	
	\subsubsection{Пример минимального документа}
	Вот пример простого документа на \LaTeX:
	
	
	\begin{lstlisting}
		\documentclass{article}
		\usepackage[utf8]{inputenc}
		\usepackage[T2A]{fontenc}
		\usepackage[russian]{babel}
		
		\begin{document}
			Привет, мир!
		\end{document}
	\end{lstlisting}
	
	\subsubsection{Преимущества перед текстовыми редакторами}
	\begin{itemize}
		\item \textbf{Автоматическая нумерация}: \LaTeX автоматически нумерует разделы, формулы, таблицы и рисунки.
		\item \textbf{Ссылки и перекрёстные ссылки}: Можно легко ссылаться на разделы, формулы и рисунки.
		\item \textbf{Библиография}: \LaTeX поддерживает управление библиографией через BibTeX или BibLaTeX.
		\item \textbf{Шаблоны}: Существует множество готовых шаблонов для статей, книг, презентаций и других типов документов.
	\end{itemize}
	
	\subsubsection{Недостатки \LaTeX}
	\begin{itemize}
		\item \textbf{Сложность обучения}: Начальное освоение \LaTeX может быть сложным для новичков.
		\item \textbf{Отсутствие WYSIWYG}: В отличие от текстовых редакторов, \LaTeX требует компиляции для просмотра результата.
		\item \textbf{Ограниченная поддержка графического интерфейса}: Большинство операций выполняется через код.
	\end{itemize}
	
	\section{Набор математических формул}
	
	\subsection{Примеры формул}
	Вот пример встроенной формулы: \( E = mc^2 \).
	
	А вот пример вынесенной формулы:
	\[
	\int_{a}^{b} f(x) \, dx = F(b) - F(a)
	\]
	
	\subsection{Математические символы}
	\LaTeX поддерживает множество математических символов, таких как:
	\[
	\alpha, \beta, \gamma, \sum, \prod, \frac{a}{b}, \sqrt{x}, \lim_{x \to \infty} f(x)
	\]
	
	\section{Построение графиков}
	
	\subsection{Использование пакета PGF/TikZ}
	PGF/TikZ — это мощный инструмент для создания векторной графики прямо в \LaTeX. Вот пример простого графика:

	\begin{center}
		\begin{tikzpicture}
			\begin{axis}[
				xlabel={\(x\)}, % Подпись оси X
				ylabel={\(y\)}, % Подпись оси Y
				xmin=-5, xmax=5, % Пределы оси X
				ymin=0, ymax=25, % Пределы оси Y
				grid=both, % Сетка
				domain=-5:5, % Область определения
				samples=100, % Количество точек
				]
				\addplot[color=blue, thick] {x^2}; % График функции
			\end{axis}
		\end{tikzpicture}
	\end{center}
	
	\section{Псевдографика}
	
	\subsection{Использование пакета listings}
	Пакет \texttt{listings} позволяет вставлять код с подсветкой синтаксиса. Пример:
	
	\begin{lstlisting}[language=Python, caption=Пример кода на Python]
		def factorial(n):
		if n == 0:
		return 1
		else:
		return n * factorial(n-1)
	\end{lstlisting}
	
	\section{Картинки}
	
	\subsection{Вставка изображений}
	Для вставки изображений используется пакет \texttt{graphicx}. Пример:
	
	\begin{center}
		\includegraphics[width=0.5\textwidth]{example-image}
	\end{center}
	
	\section{Создание презентаций}
	
	\subsection{Использование пакета beamer}
	Пакет \texttt{beamer} позволяет создавать презентации. Пример кода:
		\begin{lstlisting}
			\begin{frame}
			\begin{itemize}
				\item Первый пункт
				\item Второй пункт
				\item Третий пункт
			\end{itemize}
		\end{frame}
	\end{lstlisting}

	
	\section{Заключение}
	\LaTeX — это мощный инструмент для создания профессиональных документов. Он особенно полезен для научных работ, где требуется высококачественное оформление математических формул, графиков и таблиц. Освоение \LaTeX требует времени, но результат стоит усилий.
	
	
	
\end{document}